\documentclass{article}
\usepackage{graphicx}
\usepackage[a4paper, total={6in, 8in}]{geometry}
\usepackage{booktabs}
\usepackage{makecell}
\usepackage{siunitx}
\usepackage{amsmath}
\sisetup{
detect-all,
table-number-alignment = center,
table-figures-integer = 3,
table-figures-decimal = 4,
table-figures-exponent = 2,
exponent-product = \times,
output-exponent-marker = e
}
\usepackage{caption}
\captionsetup[table]{skip=5pt}
\usepackage{url}

\title{Effect of exotic grass cover on eucalypt seedling heights and abundances}
\author{Oliver Ekström Todreas \\ BIOS15}
\date{14 December 2025}

\begin{document}

\maketitle
\thispagestyle{empty}
\newpage




\section{Introduction}

Large datasets have the potential to uncover insights and reveal relationships that can be difficult to tease apart. The dataset analyzed in the present study is one of those, where eucalypt seedling data was collected across many plots, seasons, and environmental conditions. The aim of the present study was to determine the effect of exotic grass cover on eucalypt seedling height and abundance.




\section{Methods}

To determine the effect of exotic grass cover on eucalypt growth, two generalized linear mixed models were fitted to the dataset. Since the sampling was conducted on quadrats of land, both predictor and response variables were recorded for each quadrat. Each model had two predictor variables: the proportion of exotic annual grass cover by quadrat and exotic perennial grass cover by quadrat, both expressed in percent. Each model also accounted for the effect of quadrat property managers, allowing intercepts of the linear model to vary based on the random effect of property manager, while slopes were kept constant. The response variables were eucalypt seedling abundances per quadrat, partitioned at 50 cm in height. In other words, one model was fitted for abundances of eucalypt seedlings below 50 cm and one was fitted for abundances of eucalypt seedlings above 50 cm. The glmmTMB library was used to fit the model to the data.


\begin{align}
tall \; model &= glmmTMB(seedl.>50cm \sim ann. \; cover + peren. \; cover + (1 | prop. \; manager) \\
short \; model &= glmmTMB(seedl.<50cm \sim ann. \; cover + peren. \; cover + (1 | prop. \; manager)
\end{align}


Negative binomial errors were used when constructing the models, since the response variable was a count variable and residuals were not normally distributed. All analyses were conducted in R. Since the data for seedling abundance by size was originally partitioned into three groups (0-50 cm, 50-200 cm, 200+ cm), values for the latter two groups were summed to generate the "tall" seedlings data.




\section{Results \& Discussion}

Generally, exotic annual grass cover was higher than exotic perennial grass cover across quadrats, at 12.4\% and 3.71\% respectively. There were on average 1.21 tall eucalypt seedlings recorded per quadrat and 0.662 short eucalypt seedlings recorded per quadrat. Variation in grass cover scaled with grass cover. The variation in abundance was lower for tall eucalypt seedlings than it was for short eucalypt seedlings, for which standard deviations were 3.17 and 4.59 respectively (Table 1).

\begin{table}[ht]
\centering
\caption{Summary statistics of predictor and response variables by quadrat}
\label{tab:summary-stats}
\begin{tabular}{lcccc}
\toprule
Statistic &
\makecell{Exotic annual\\grass cover (\%)} &
\makecell{Exotic perennial\\grass cover (\%)} &
\makecell{Eucalypt seedling\\abundance $>$ 50 cm} &
\makecell{Eucalypt seedling\\abundance $<$ 50 cm} \\
\midrule
Mean & 12.4 & 3.71 & 1.21 & 0.662 \\
SD   & 15.4 & 8.97 & 3.17 & 4.59 \\
\bottomrule
\end{tabular}
\end{table}

Around half of the variance in responses was attributed to the grouping variable for both models, in this case the property manager. Specifically, 42.8\% of the variance in short seedling abundance and 58.7\% of the variance in tall seedling abundance was attributed to the property manager (Table 2).

\begin{table}[ht]
\centering
\caption{Unadjusted intraclass correlation coefficients (ICC) for eucalypt seedling abundance by height per quadrat}
\label{tab:icc-eucalypt}
\begin{tabular}{l c}
\toprule
Eucalypt seedling abundance model & Unadjusted ICC \\
\midrule
Short ($<50$ cm) / quadrat & 0.428 \\
Tall ($>50$ cm) / quadrat & 0.587 \\
\bottomrule
\end{tabular}
\end{table}

Given that a large share of the variance in response was attributed to the random effect of property manager, it was unsurprising to see a large span in intercepts generated by the models. For short seedlings, random-effect intercepts ranged from 0.065 seedlings to 2.28 seedlings per quadrat, and for tall seedlings, they spanned 0.051 to 3.1. The interquartile range covered over 0.5 seedlings for short seedlings and 1 seedling for tall seedlings (Table 3). Both the range of intercepts and the interquartile range were larger for the tall seedling model than the short seedling model, corroborating the trend observed in the model ICCs. In other words, the variation attributed to the random effect of property manager was on display both in the ICC estimates and the intercept values.

This leaves only about half of the variation in seedling abundances to be attributed to the exotic grass cover. According to the model for tall eucalypt seedlings, going from 0-100\% exotic annual grass cover resulted in an increase from 0.45 to 0.77 seedlings per quadrat, a 71\% increase (Figure 1A). For the perennial grass, going from 0-100\% cover represented a decrease of 95\%, from 0.54 to 0.025 seedlings (Figure 1B). For the model for short eucalypt seedlings, going from 0-100\% annual cover resulted in a decrease of 28\% in seedling abundance, going from 0.24 to 0.17 seedlings per quadrat (Figure 1C). Lastly, for perennial cover, there was an expected decrease of 98\% seedling abundance, from 0.26 to 0.0041 seedlings per quadrat (Figure 1D).

\begin{figure}[htp]
\centering
\includegraphics[width=10cm]{outputs/fig1.pdf}
\caption{Eucalypt seedling abundance by height and exotic grass cover. What is clear in the figure is that the quadrats with more than a few single seedlings represent a small share of the total number of quadrats. While hard to visually identify, there was a positive relationship between exotic grass cover and eucalypt seedling abundance for exotic annual grass and tall seedlings (A), and a negative relationship in the rest (B-D). Exotic perennial grass had a larger effect on percent change in eucalypt seedling abundance for both tall and short seedlings compared to exotic annual grass. Note that one data point representing a quadrat with 78 seedlings was not plotted, but was still included in the model.}
\label{fig:plts}
\end{figure}

Overall, the models suggest that exotic perennial grass cover has a larger negative effect on eucalypt seedling abundance than exotic annual grass for both short and tall seedlings. The only positive relationship between exotic grass cover and eucalypt seedling abundance was observed for the effect of exotic annual grass cover on tall seedling abundance. 


\begin{table}[ht]
\centering
\caption{Summary of intercepts of random-effect groups}
\label{tab:re-var-growthform}
\begin{tabular}{l c c c c c}
\toprule
Eucalypt seedling abundance model & Min. & 1Q & Med & 3Q & Max. \\
\midrule
Short ($<50$ cm) & 0.065 & 0.104 & 0.184 & 0.638 & 2.280 \\
Tall ($>50$ cm) & 0.051 & 0.137 & 0.654 & 1.890 & 3.100 \\
\bottomrule
\end{tabular}
\end{table}




\section{Appendix}

The data, code, and figures used to produce this report can be downloaded at the link below:
\url{https://github.com/otodreas/BIOS15_PracticeExam} in the /scripts folder.

\end{document}
